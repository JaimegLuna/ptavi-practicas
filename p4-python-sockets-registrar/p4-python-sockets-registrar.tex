\documentclass[11pt,a4paper]{article}

\usepackage[utf8]{inputenc}
\usepackage{graphics}
\usepackage{url}
\usepackage{amssymb} % Simbolos matematicos

\newcommand{\finejercicio}{
  \begin{footnotesize}
    [Al terminar el ejercicio es recomendable hacer \texttt{commit} de los ficheros modificados]
  \end{footnotesize}
}

\newcommand{\finpractica}{
  \begin{footnotesize}
    [Al terminar la práctica, realiza un \texttt{push} para sincronizar tu repositorio GitHub]
  \end{footnotesize}
}

%\renewcommand{\finejercicio}{}
%\renewcommand{\finpractica}{}


\ProcessOptions

\begin{document}


\title{Práctica 4 - Python, sockets UDP y Registrar SIP}
\author{Protocolos para la Transmisión de Audio y Vídeo en Internet}
\date{Versión 5.0.5 – 20.10.2014}


\maketitle

Nota: Esta práctica se puede entregar para su evaluación como parte de la nota de prácticas, pudiendo obtener el estudiante hasta un punto. Para las instrucciones de entrega, mira al final del documento. Para la evaluación de esta entrega se valorará el correcto funcionamiento de lo que se pide, el seguimiento de la guía de estilo de Python y el correcto uso (y entrega) con git en GitHub.

\section{Introducción}

\texttt{SocketServer} es un módulo en Python que simplifica la tarea de implementar servicios en Internet. Aunque hay cuatro tipos diferentes de servidores básicos, nosotros en esta práctica utilizaremos solamente \texttt{UDPServer} que utiliza datagramas, son paquetes discretos que pueden llegar desordenados – incluso se pueden perder por el camino (nótese que esto es en contraposición con \texttt{TCPServer} que trabaja con flujos TCP). \texttt{UDPServer} trabaja de manera secuencial, lo que significa que las peticiones serán atendidas de una en una, por lo que tendrán que esperar si hay una petición en proceso.

\section{Objetivos de la práctica}

\begin{itemize}
  \item Manejar SIP de manera sencilla.
  \item Crear un esquema cliente-servidor en Python.
\end{itemize}

\section{Conocimientos previos necesarios}

\begin{enumerate}
  \item Nociones de Python (las de las primeras prácticas) y de orientación a objetos.
  \item Nociones de SIP (las vistas en clase de teoría)
\end{enumerate}

Tiempo estimado (para un alumno medio): 10 horas

\section{Ejercicios}

\begin{enumerate}

  \item Con el navegador, dirígete al repositorio \texttt{ptavi-p4} en la cuenta del profesor en GitHub\footnote{\url{http://github.com/gregoriorobles/ptavi-p4}} y realiza un \texttt{fork}\footnote{Tienes instrucciones de cómo realizar un \texttt{fork} en \url{https://guides.github.com/activities/forking/}.}, de manera que consigas tener una copia del repositorio en tu cuenta de GitHub. Clona el repositorio que acabas de crear a local para poder editar los archivos. Trabaja a partir de ahora en ese repositorio, sincronizando (commit) los cambios que vayas realizando según los ejercicios que se comentan a continuación.

  Como tarde al final de la práctica, deberás realizar un \texttt{push} para subir tus cambios a tu repositorio en GitHub, aunque si quieres que tus compañeros de equipo puedan ver el código antes, se recomienda hacerlo con anterioridad.

  \item Estudia el código de un sencillo cliente UDP en \texttt{client.py} que encontrarás en tu repositorio local. Fíjate en que:
  \begin{enumerate}
    \item Importa el módulo \texttt{socket}.
    \item Inicializa varias \emph{variables} constantes (al ser constantes, están en mayúsculas).
    \item Crea un \texttt{socket}.
    \item Especifica opciones para el \texttt{socket}.
    \item Se conecta con un servidor.
    \item Envía por el \texttt{socket} con el método \texttt{send()} y lee del \texttt{socket} con \texttt{recv()}.
    \item Al recibir con \texttt{recv()}, indicamos el valor del \emph{buffer} en bytes.
    \item Cerramos la conexión con el método \texttt{close()}.
  \end{enumerate}


  \item Estudia el código de \texttt{server.py}, que implementa un sencillo servidor de eco basado en UDP. Fíjate en que:
  \begin{enumerate}
    \item Importa el módulo \texttt{SocketServer}\footnote{\url{http://www.python.org/doc/2.5.2/lib/module-SocketServer.html}}.
    \item Tenemos una única clase que manejará las peticiones.
    \item Esta clase hereda de una clase \texttt{DatagramRequestHandler} que hay en el módulo \texttt{Socket}.
    \item La clase no tiene constructor \texttt{\_\_init\_\_}; utiliza el de la clase padre.
    \item La clase sólo tiene un método, llamado \texttt{handle()}.
    \item El método \texttt{handle()} se ejecuta cada vez que recibimos una petición al servidor.
    \item \texttt{self.wfile} y \texttt{self.rfile} son los atributos que abstraen el \texttt{socket} (como si fuera un fichero):
    \begin{itemize}
      \item Podemos leer del mismo con \texttt{self.rfile.read()}.
      \item Podemos escribir en el mismo con \texttt{self.wfile.write()}.
    \end{itemize}
    \item Un programa principal, donde se instancia la clase \texttt{EchoHandler}, indicando la IP y el puerto donde se deja al servidor escuchando en un bucle infinito.
  \end{enumerate}

  \item Cambia el \emph{script} del cliente para que:
  \begin{itemize}
    \item Se pase como parámetro al \emph{script} la IP y el puerto del servidor, así como a continuación el mensaje que se ha de enviar. Así, para ejecutar el cliente, deberíamos hacer:
    \begin{verbatim}
	$ python client.py ip puerto linea
    \end{verbatim}
  Una llamada de ejemplo podría ser: 
\begin{verbatim}
python client.py 127.0.0.1 5060 eco eco, soy yo
\end{verbatim}
  \end{itemize}

\finejercicio


  \item Modifica el \emph{script} del servidor para que:
  \begin{itemize}
    \item Imprima por pantalla la IP y el puerto del cliente (esta información viene en el atributo \texttt{client\_address} en forma de tupla\footnote{Las tuplas son listas especiales, ya que son inmutables. Se definen con paréntesis, no con corchetes.}).
    \item Se pase el puerto al que ha de escuchar como parámetro al \emph{script}.
  \end{itemize}

\finejercicio

  \item Modifica los \emph{scripts} anteriores para tener un cliente que envíe mensajes de registro SIP y un servidor \texttt{Registrar} SIP. Cada vez que el cliente le mande una línea con el método \texttt{REGISTER} (en mayúsculas), el servidor guardará la dirección registrada y la IP en un diccionario. Cambia el nombre de la clase del servidor de \texttt{EchoHandler} a \texttt{SIPRegisterHandler}. La petición SIP del cliente tendrá esta pinta:
    \begin{verbatim}
REGISTER sip:luke@polismassa.com SIP/2.0\r\n\r\n
    \end{verbatim}
  El servidor deberá responder con un mensaje de este estilo:
\begin{verbatim}
SIP/2.0 200 OK\r\n\r\n
\end{verbatim}
  El cliente se deberá seguir ejecutándose desde línea de comando, ahora de la siguiente manera:
    \begin{verbatim}
		$ python client.py ip puerto register luke@polismassa.com
    \end{verbatim}

\finejercicio

  \item Añade funcionalidad al cliente y al servidor para ofrecer la posibilidad de darse de baja a un usuario. Para ello, hay que añadir una cabecera \texttt{Expires} (cuyos valores vienen dados en segundos). En caso de que el valor de la cabecera \texttt{Expires} sea 0, el usuario será borrado del diccionario de registro; en cualquier otro caso, siempre que el tiempo sea positivo, el valor de \texttt{Expires} será el tiempo de expiración en el servidor. En este ejercicio no hace falta considerar funcionalidad de servidor para cuando el registro de un cliente caduca. La petición del cliente tendrá una pinta parecida a ésta:
  \begin{verbatim}
REGISTER sip:luke@polismassa.com SIP/2.0\r\n
Expires: 0\r\n\r\n
  \end{verbatim}
El servidor deberá borrar al usuario del diccionario y responder con un
\begin{verbatim}
SIP/2.0 200 OK\r\n\r\n
\end{verbatim}
El cliente se deberá seguir ejecutando desde línea de comando, ahora de la siguiente manera:
\begin{verbatim}
$ python client.py localhost 2500 register luke@polismassa.com 3600
\end{verbatim}
	donde 3600 es un ejemplo del tiempo de expiración y será el valor de la cabecera \texttt{Expires}.
En caso de que no se introduzcan los valores necesarios, se imprimirá el siguiente mensaje:
\begin{verbatim}
Usage: client.py ip puerto register sip_address expires_value
\end{verbatim}

\finejercicio

  \item  Modifica el servidor para añadir el método \texttt{register2file} en el que se implemente la siguiente funcionalidad: cada vez que un usuario se registre o se dé de baja, se imprimirá en el fichero \texttt{registered.txt} lo siguiente: en una primera línea los campos separados por tabuladores, y en sucesivas líneas los valores de cada uno de los usuarios registrados, siguiendo el siguiente formato\footnote{Los espacios antes y después de los tabuladores son para mejorar la visualización del guión de prácticas, por lo que no deberían incluirse en el fichero}:
\begin{verbatim}
		User \t IP \t Expires
		luke@polismassa.com \t localhost \t 2013-10-23 10:37:12
		papa@darthwader.com \t localhost \t 2013-10-23 10:21:15
\end{verbatim}
	Para los formatos de tiempo, se recomienda utilizar el módulo time, en particular:
\texttt{time()}, que devuelve los segundos desde el 1 de enero de 1970,
\texttt{gmtime()}, que toma los segundos desde el 1 de enero de 1970 y te lo devuelve en una tupla,
y \texttt{strptime()}, que representa el tiempo en un \texttt{string}.
	Así, 
\begin{verbatim}
time.strftime('%Y-%m-%d %H:%M:%S', time.gmtime(time.time()))
\end{verbatim}

	devolverá un \texttt{string} con la hora GMT actual. Y
\begin{verbatim}
		time.strftime('%Y-%m-%d %H:%M:%S', time.gmtime(1233213))
\end{verbatim}
	devolverá un \texttt{string} con la hora GMT del segundo 1.233.213 desde el 1 de enero de 1970.
En este ejercicio, se ha de implementar funcionalidad para que el servidor \texttt{Registrar} gestione la caducidad de los usuarios registrados.

\finejercicio

  \item  Documenta tu código con \texttt{docstrings}\footnote{La recomendación para \texttt{docstrings} en Python se puede encontrar en \url{http://legacy.python.org/dev/peps/pep-0257/}.}. Comprueba asimismo que los nombres de las variables siguen las indicaciones de PEP8. Ten en cuenta que el programa \texttt{pep8} no comprueba ninguna de estas dos circunstancias.

\finejercicio

  \item Realiza una captura con \texttt{wireshark} (interfaz \texttt{lo}) con las siguientes interacciones:
  \begin{enumerate}
    \item El cliente \texttt{marty.mcfly@sk8ing.com} se registra. Tiempo de expiración: 5.
    \item El cliente \texttt{doc@delorean.com} se registra. Tiempo de expiración: 3600.
    \item (Se dejan pasar unos segundos, más de cinco).
    \item El cliente \texttt{doc@delorean.com} se da de baja.
    \item El cliente \texttt{marty.mcfly@sk8ing.com} se da de baja.
  \end{enumerate}
	Investiga tu captura. Comprueba, en particular, que puedes ver el intercambio de mensajes y que todo va en texto claro por la red. Guarda la captura como \texttt{register.libpcap} y añádela al repositorio.

\finejercicio

\finpractica

\end{enumerate}

\section{¿Qué deberías tener al finalizar la práctica?}

La entrega de práctica se deberá hacer antes del lunes 27 de octubre de 2014 a las 23:55. Para entonces, se debe: 

\begin{enumerate}
  \item Tener un repositorio git en GitHub con:
  \begin{itemize}
    \item 2 módulos Python y la captura realizada con \texttt{wireshark} (y únicamente estos tres ficheros):
    \begin{itemize}
      \item \texttt{server.py}
      \item \texttt{client.py}
      \item \texttt{register.libpcap}
    \end{itemize}
    \item 1 clase \texttt{SIPRegisterHandler} (en \texttt{server.py}) con los métodos:
    \begin{enumerate}
      \item \texttt{handle}
      \item \texttt{register2file}
    \end{enumerate}
    \item 4 ficheros adicionales (además de \texttt{.git}): \texttt{README.md}, \texttt{LICENSE}, \texttt{check-p4.py} y \texttt{.gitignore}.
  \end{itemize}
\end{enumerate}

Se ha de tener en cuenta las siguientes consideraciones:
\begin{itemize}
  \item Se valorará que al menos haya diez \texttt{commits} realizados.
  \item Se valorará que el código entregado siga la guía de estilo de Python (véanse PEP8 y PEP257).
  \item Se valorará que los programas se invoquen correctamente y que muestren los errores correctamente, según se indica en el enunciado de la práctica.
  \item Se valorará que el equipo al que pertenece el alumno siga la guía de estilo
de Python, tenga código legible y bien estructurado.
\end{itemize}

Se puede comprobar la correcta entrega de la práctica utilizando el programa \texttt{check-p4.py}, bajándoselo de Moodle. Este programa se ejecuta desde la línea de comandos de la siguiente manera:
\begin{verbatim}
	$ python check-p4.py login_lab
\end{verbatim}


donde \texttt{login\_lab} es tu nombre de usuario en los laboratorios docentes de Linux del GSyC. El programa comprueba que se han entregado los ficheros que se solicitan (y sólo esos), y parcialmente si se sigue la guía de estilo PEP8.

\vspace{1cm}

\footnotesize{Para el final: Si a estas alturas todavía tienes ganas, puedes indicar en el Moodle el tiempo que te ha llevado realizar esta práctica. Esto te lleva menos de un minuto y nos sirve a los profesores para hacernos una idea del esfuerzo que os lleva. ¡Gracias!}

\end{document}
